\documentclass[a4paper, 12pt]{article}
\usepackage{multicol}
% *********  Set some sensible page margins  ***********
\setlength{\oddsidemargin}{0in} \setlength{\evensidemargin}{0in}
\setlength{\textwidth}{6.2in}
\setlength{\topmargin}{-0.3in} \setlength{\textheight}{9.8in}
\title {Departmental Reports - Standards}
\author {Nigel Hardy}
\date{30 June 1994} % Leave this out to generate date automatically

\begin{document}
\maketitle
\begin{multicols}{2}
\section{Purpose of this document}

This document lays down the standards and procedures
for the production, quality assurance and archiving of
technical reports within the department.


\section{Management}

A “Technical Report Manager” is appointed by
the Head of Research. This person is responsible
for
\begin{enumerate}  

\item Issuing unique report references (see section~\ref{sec:docrefs})
\item Quality assurance on matters of presentation (see Section~\ref{sec:presstand})
\item Assuring that reports pass through a quality assurance process which considers their contents (see Section~\ref{sec:contstand}
\item Acceptance, archiving and distribution of reports which have passed both quality as assurance processes (see Section~\ref{sec:arch})
\end{enumerate}

\section{Document references}
\label{sec:docrefs}

All reports have a reference of the form
\hspace{1cm}UWA-DCS-yy-xxx
\newline
where yy is the year of original production and xxx is a unique number for the report

\section{Presentation Standards}
\label{sec:presstand}
All reports will be produced using the deptrept \LaTeX style file.
\newline
\indent In general terms this defines:
\begin{enumerate}
\item material for the front page which includes:
\begin{enumerate}
\item the title;
\item he report reference;
\item the date of the version;
\item the version number;
\item the report status;
\item the address of the department and a copyright notice.
\end{enumerate}
\item page, section and similar styles for the body of the document.
\end{enumerate}
A proform source file (report-proforma.tex) to
accompany the LaTeX style is available and gives additional guidelines.

It is not intended that tight control be exercised over
the fine detail of presentation. Attempts to alter the
general appearance of the documents from that layed
down will be rejected. It is recognised that on rare
occasions a special document may need to deviate.

\section{Bibilography}
The plain bibliography style should be used or followed. This will result in bibliographies 
as shown in \cite{1}
\section{Content Standards}
\label{sec:contstand}
Before promotion from draft to release or restricted
status (see Section ) a report must pass a quality assurance process which examines its content. A change of
status form, available from the Technical Report Manager, must be completed by the person approving a report.

Three methods are available, reflecting the circumstances in which a report is produced.

\subsection{Postgraduate Students}
For reports written by postgrad students (as part of
their directed study), the report should be approved by
the supervisor.
\subsection{Research Associates and Assistants}
For reports produced by RAs working on funded
projects (as part of that project), the report should be
approved by a principal investigator.
\subsection{Staff}
For reports produced by members of staff and for those
produced by postgraduates or RAs other than as part of
their main project, at least one member of the academic
staff should read and approve it.
\section{Archiving and Distribution}
\label{sec:arch}
Writers of reports are responsible for keeping the source
of the report. The Technical Report manager will receive and archive postscript versions.
\section{Report Status}
The staus of a report will be one of:
\begin{itemize}
\item draft
\item restricted
\item release
\end{itemize}
A report starts as draft. It is not for distribution
beyond those concerned with producing or reviewing
it. When both presentation and contents have the QA
procedures it can progress to release or restricted. A
report with restricted status is assumed to contain confidential information and can only be made available
by its author(s) or by the supervisor or principle investigator responsible for its production. A report with
release status can be freely distributed by anybody, this
includes people outside the department passing it on to
others.
\section{Report Status}
\label{sec:report}
When a report is approved for promotion from draft
status it becomes version 2.1. Draft versions will therefore be 1.x.
When a report is approved for promotion from draft
status it becomes version 2.1. Draft versions will therefore be 1.x.

After the initial release or restricted version, the author(s) may work further on the document and wish to
produce a new version. This should be done by working
on documents whose status returns to draft and which
have version numbers 2.2, 2.3 etc. At some stage the
achieve release or restricted status with a version number of 3.1. This process can continue through major
version numbers. A new major version need not have
the same status as its predecessor, i.e. a restricted document could become release in a later version.

Old release versions of reports will remain available
for distribution. If copies are requested, both the current and any specific older version requested will be
sent.
\section{Summary}
The "lifecycle" of a report is as follows.
\begin{enumerate}
\item The author thinks (or is told) of something to write
about.
\item The author sends the title to the Technical Report
Manager and requests a report reference.
\item The author seeks an appropriate person to check the
contents for quality.
\item That person (eventually) fills in a change of status
form.
\item The author passes the change of status form to the
Technical Report Manager, along with the pathname of the postscript of version 2.1
\item The Technical Report Manager checks that the document meets presentation standards. Changes may
be requested. The author must keep on providing
updated postscript files which say version 2.1. The
author should be able to do this assuming that 2.1
is created from the latest 1.x with no change of content. It is unlikely that any changes required will
invalidate the change of status form, but the manager should be aware of this possibility.
\item The Technical Report Manager archives and distributes the postscript.
\item Updates progress via the same steps, omitting 1, 2
and possibly 3. All updates must be checked for
both content and presentation.
\end{enumerate}
\begin{thebibliography}{9}
\bibitem{1}
Nigel Hardy. Departmental reports - stan-
dards.   Technical report, Department of
Computer  Science,  University  of  Wales,
Aberystwyth, 1994.
\end{thebibliography}
\end{multicols}
\end{document}
