\documentclass{article}
\usepackage[left=1.25in,right=1.25in,top=1in,bottom=1in]{geometry}

\title{Practical 8: An exercise in \LaTeX}
\author{Michael Clarke}
\date{\today}

\begin{document}
\maketitle

\section{Objectives}
In this practical you are given a text document, and asked to mark it up and
process it using \LaTeX\ so that it looks something like the supplied PDF
document. These two documents are available from blackboard.

You will need to download the .txt file to your CS10110 directory (or somewhere else suitable).

The exercise is for you to produce output looking as much like the provided PDF 
as possible. You should note the following:

\begin{itemize}
\item The \verb|article| document class was used.
\item There is a \verb|twocolumn| option to the document class (see The Not So Short introduction to \LaTeX2e).
\item The document contains a number of {\em See Section x} internal cross references. you will need \verb|label/ref| pairs to implement these.
\item You may find the \verb|quotation| environment helpful.
\item For the bibliography, the smart way to do it is using something called BibTeX, but we haven't time to go into that in this module. Investigate \verb|theBibliography| instead. Don't worry too much about this though, if you get lost.
\end{itemize}

You should produce a source file called doc.tex. Initially, just copy the text file to this file; it will be the basis for your document.

You should produce PDF output in a file called \verb|doc.pdf|.

\section{Suggestions}
\begin{enumerate}
\item An example of the output required is available on-line, linked from Blackboard.
\item The table need not be a floating fable as illustrated in the example. You do not, therefore, require a ``List of Tables''.
\item It is probably best to start with a very simple document, check that you can process it and view the output, and then build up towards the final document.
\item Few people actually remember all the commands for the headers of a \LaTeX document, they just copy and paste them from a previous document. So, to get you started, you can use the template.tex file, also available on blackboard.
\end{enumerate}
\end{document}
