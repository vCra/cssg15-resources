\documentstyle{article}
\title{ISDN to improve Health Care provision in rural areas}
\author{Neal Snooke\\
 Document Originated : May 5, 1993}
\begin{document}
\maketitle

\section{Introduction}
Inexpensive dial-up digital data communications over the public telephone
network is finally becoming a reality in Rural Wales.  With this 
communications platform
we can look forward to many new developments to improve the services 
available to, and the working practices of many health care professionals. 

\section{Specific Needs of Rural Areas}
Until recently similar polices and facilities have been implemented for
healthcare provision in rural and non rural parts of the country.  
Although advanced IT systems can provide benefits in all areas the 
specific problems assiciated with low population density and 
geography provide the potential for much greater gains in quality
of service.  Some examples of these difficulties are large distances for patients to travel to:
\begin{itemize}
\item to GPs, 
\item Community Hospitals (CH) 
\item District General Hospitals (DGH).  
\end{itemize}
Often specific specialist 
consultation is not available within reasonable travel time, although
this can apply also to non rural areas to a lesser degree.  
Medical staff at all levels ofen feel isolated from collegues both for
consultation and for training purposes.  Many institutions have
part time visiting staff who are only available periodically at
each locality.  Examples being the radiologist, dermatologist etc.
 
\section{Can ISDN be used effectively ?}

The narrowband ISDN that is being provided in Wales today is
well suited for transmission of some types of information (
Voice/audio, Textual) in any situation.  Other multimedia 
forms (Images, Video) can be provided with some restrictions. 
 
\end{document}
